\documentclass{article}
\title{C++ Notes}
\author{Jonathan Bowden}

\usepackage[margin=2cm]{geometry}
\usepackage[UKenglish]{isodate}
\usepackage{listings}
\usepackage{xcolor}

\definecolor{cgreen}{RGB}{0,128,0}
\definecolor{cred}{RGB}{163,21,21}

%\part{The Part Title}
%\section {The Section Title}
%\subsection {Subsection}
%\subsubsection {SubSubSection}
%\paragraph {Paragraph}
%\subparagraph {SubParagraph}

\begin{document}


\lstset{language=C++,
	     breaklines=true,
                basicstyle=\ttfamily,
                keywordstyle=\color{blue}\ttfamily,
                stringstyle=\color{cred}\ttfamily,
                commentstyle=\color{cgreen}\ttfamily,
                morecomment=[l][\color{magenta}]{\#}
}


\maketitle

\section{Operators}

\begin{itemize}
\item Variable Address

\begin{lstlisting}
	cout << &A << endl
\end{lstlisting}

\item Joining Strings

\begin{lstlisting}
	string combinedStrings = x + `` ''  + y; // just use plus to combine string
\end{lstlisting}

\item{Incrementing}

\begin{lstlisting}
	cout << "INCREMENTATIAON EXAMPLES" << endl;
	int d = 1;
	cout << d++ << endl; // returns one because it outputs d first, then increments (POST-INCREMENTATION)
	cout << ++d << endl; // increments first (PRE-INCREMENTATION)


	cout << "DECREMENTATIAON EXAMPLES" << endl;
	int e = 1;
	cout << e-- << endl; // returns one because it outputs d first, then decrements (POST-DECREMENTATION)
	cout << --e << endl; // decrements first (PRE-DECREMENTATION)
\end{lstlisting}

\item And (conjunction)
\begin{lstlisting}
	cout << ((7 < 5) && (5 != 10)) << endl;
\end{lstlisting}

\item OR (disjunction)

\begin{lstlisting}
	cout << ((7 < 5) || (5 != 10)) << endl;
\end{lstlisting}

\item{Bitwise Operators}
\begin{lstlisting}
	/*
	Bitwise AND - &
	Bitwise OR - |
	Bitwise NOT - ~ (tilde)
	Bitwise XOR - ^ (caret)
	Bitwise left shift - << 
	Bitwise right shift - >>
	*/
\end{lstlisting}

\item{}
\begin{lstlisting}

\end{lstlisting}

\item{}
\begin{lstlisting}

\end{lstlisting}

\item{}
\begin{lstlisting}

\end{lstlisting}

\item{}
\begin{lstlisting}

\end{lstlisting}

\item{}
\begin{lstlisting}

\end{lstlisting}





\section{Logic}

\item{IF-THEN-ELSE}
\begin{lstlisting}
	if (a > b) {
		cout << a << " > " << b << endl;
	}
	else if (a < b)
		cout << a << " < " << b << endl;
	else
		cout << "conditions not met" << endl;

\end{lstlisting}

\item{SWITCH-CASE}
Need to remember the break; command at the end of each case, or C++ will execute sequentially.
\begin{lstlisting}
	int x = 0;
	
	switch (x) //executes all code after condition is met
	{
	case 0:
		cout << "code here when case is 0" << endl;
		break;
	case 25:
		cout << "code here when case is 25" << endl;
		break;
	case 50:
		cout << "code here when case is 50" << endl;
		break; // stops the rest of the swtich code being executed
	default:  
		cout << "code here when value is nothing else" << endl;
	}

\end{lstlisting}

\item{CONDITIONAL OPERATOR - ?}
\begin{lstlisting}
	string message  = (a > b) ? "a > b" : "a <= b";

	cout << ((a > b ? a : b)) + 10 << endl; // add 10 to the higher number
\end{lstlisting}


\item{For Loops}
\begin{lstlisting}

	for (init; condition; inc/dec)
	
	for (int i = 0; i < 5; i++)
	{
		cout << "HELLO" << endl;
	}
	
	
	int arr[100];

	for (int i = 0; i < 100; i++)
	{
		arr[i] = i;
		cout << arr[i] << endl;
	}

\end{lstlisting}


\item{Do Loops}
\begin{lstlisting}

	while (--i) // putting the increment before the variable, checks the "next" condition before executing loop
	{
		cout << i << endl;
	}
	
	
	int arr[sizeofarray];
	
	while(i < sizeofarray)
	{
		arr[i] = 10 * i;
		cout << arr[i++] << endl; // First send to the ouput, then increment
	}
	
	
	do
	{
		cout << "lala";
	} while (i); //check condition at end

\end{lstlisting}



\item{Continue and Next}

Continue keeps the loop going, break does not.

\begin{lstlisting}
	
	
	for (int i = 1; i <= 10; i++) // i = 2
	{
		//if (i == 5)
		//	continue;  // everything after the continue won't be executed, but the loop won't be stopped.

		//if (i == 5)
		//	break;  // everything after break won't be executed and the loop is stopped.

		for (int j = 1; j <= 10; j++) // j = 1
		{
			if (j == 5)
				break; //exits the loop, continue just skips the 5th one
			cout.width(4);
			cout << i * j;
		}


		cout << endl;
	}

	

	for (int i = 1, j = 1; i <= 10; i++)
	{
		cout.width(4);
		cout << i * j;

		if (i == 10)
		{
			j++;
			i = 0;
			cout << endl;
		}

		if (j == 10 + 1) //add plus one to see the 10th row
			break;
	}


\end{lstlisting}




\item{}

\begin{lstlisting}

\end{lstlisting}

\end{itemize}






\section{Variables}

\begin{itemize}

\item{Arrays}
\begin{lstlisting}
	int arr[4];
	arr[0] = 10;

	int biarr[3][4] = { 0 };
	int triarr[2][3][2];

	// the first array item represents the address of the entire array as well:
	cout << "Array address: " << &arr << endl;
	
	
\end{lstlisting}

\end{itemize}







\section{Code snippets}

\begin{itemize}

\item{Data Validation}
\begin{lstlisting}
	
bool isValid(string error_msg)
{
	if (cin.rdstate()) // state is wrong when not equal to zero
	{
		cin.clear();
		cin.ignore(numeric_limits<streamsize>::max(), '\n');
		system("cls");
		initMenu();
		cout << error_msg << endl;
		return false; // return leaves the function.
	}
	return true;
}

// This is used in the main code as follows:

	do { cin >> a; } while (!isValid("The input is invalid"));
	areaSquare(a);

// Can also use an overload function without an error message:

bool isValid()
{
	if (cin.rdstate()) // state is wrong when not equal to zero
	{
		cin.clear();
		cin.ignore(numeric_limits<streamsize>::max(), '\n');
		system("cls");
		initMenu();
		return false; // return leaves the function.
	}
	return true;
}

// Used as follows:

	do { cout << "Enter the radius: " << endl; cin >> r; } while (!isValid());
	areaCircle(r);

\end{lstlisting}



\begin{lstlisting}

\end{lstlisting}

\begin{lstlisting}

\end{lstlisting}

\begin{lstlisting}

\end{lstlisting}

\begin{lstlisting}

\end{lstlisting}



\end{itemize}



\end{document}